
\documentclass[12pt]{article}
\usepackage{amsmath}
\DeclareMathOperator*{\argmin}{arg\,min} % thin space, limits underneath in displays
\DeclareMathOperator*{\argmax}{arg\,max} % thin space, limits underneath in displays
\newtheorem{thm}{Theorem}
\usepackage{amssymb}
\usepackage{amsfonts}
\usepackage{mathrsfs}
\usepackage{bm}
\usepackage{indentfirst}
\setlength{\parindent}{0em}
\usepackage[margin=1in]{geometry}
\usepackage{graphicx}
\usepackage{setspace}
\doublespacing
\usepackage[flushleft]{threeparttable}
\usepackage{booktabs,caption}
\usepackage{float}
\usepackage{graphicx}
\usepackage[sort,comma]{natbib}
\usepackage[hidelinks]{hyperref}

\usepackage{import}
\usepackage{xifthen}
\usepackage{pdfpages}
\usepackage{transparent}

\newcommand{\incfig}[1]{%
\def\svgwidth{\columnwidth}
\import{./figures/}{#1.pdf_tex}
}




\title{IS-LM}
\author{}
\date{}



\begin{document}
\maketitle
CH9

\section{Consumption and AD}
\subsection{AD}
Assumption:
\begin{equation*}
AD = C + I + G + NX
\end{equation*}
\begin{itemize}
\item For simplicity, we assume no gov't and international trade, i.e., 
		$ G = NX = 0 $.
\item $ Y = C + I $
\item {\textbf {Income (Y) can be used for either C or S (savings)}}
\end{itemize}


Consumption function:
\begin{equation}
C =  \overline{C} + cY, \quad  \overline{C}>0, c \in (0,1) \label{eqn:consumption}
\end{equation}
where $ c $ is marginal propensity of consumption MPC.  And $  \overline{C} $ is the
autonomous consumption (consumption level HHs needed even there is no income, e.g., 
food).
If c = 0.2, for additional 1 dollar income, you will spend 0.2 on consumption. 

Saving:
\begin{equation}
S = Y - C \label{eqn:saving}
\end{equation}
Plug in equation \eqref{eqn:consumption} in to the above
\begin{align*}
S &= Y -  \overline{C} - cY =  -  \overline{C} + (1 - c)Y\\
s + c &= 1, \text{ s: marginal propensity of saving MPS }\\
S &=  -  \overline{C} + sY
\end{align*}


\subsection{AD and output (Y)}
Equilibrium appears when the total output $ Y $ equals to the aggregate demand.
\begin{equation*}
Y = AD
\end{equation*}
We use {\textbf {unplanned investment (UI) to measure the difference between the 
AD and actual output level.}}
\begin{align*}
UI &= Y - AD\\
UI > 0, &\text{ if  }Y>AD\\
UI < 0, &\text{ if  }Y<AD\\
\end{align*}

Put output $ Y $ on the horizontal axis, and $ AD $ on the vertical axis.
\begin{figure}[ht]
    \centering
		\scalebox{.4}{\incfig{ad=y}}
    \label{fig:ad=y}
\end{figure}



Now we add G and NX to our model. But we assume that $ I, G, NX $ are constant.
They are exogenous. We take then as given.
We also introduce tax ($ T $) and lump sum transfer ($ TR $) from gov't to our model.
They are also exogenous.

{\textbf {Disposable Income (YD)}} measures HH's income after tax and government 
transfer.
\begin{equation*}
YD = Y - T + TR
\end{equation*}
Hence, HH's consumption is decided by the $ YD $. We can rewrite the consumption function
\begin{equation}
C =  \overline{C} + c(YD) =  \overline{C} + c(Y - T + TR)
\end{equation}
Let $  \overline{I},  \overline{G},  \overline{NX},  \overline{T},  \overline{TR} $
be the exogenous variables.
The aggregate demand could be written as
\begin{align*}
AD &= C + I + G + NX\\
&=  \overline{C} + c(Y -  \overline{T} + \overline{TR}) +  \overline{I} +  \overline{G}
 +  \overline{NX}\\
&= [ \overline{C} - ( \overline{T}  -  \overline{TR}) +  \overline{I} +  \overline{G}
 +  \overline{NX}] + cY\\
AD &=  \overline{A} + cY
\end{align*}
where $  \overline{A} = [ \overline{C} - ( \overline{T}  -  \overline{TR}) +  \overline{I} +  \overline{G} +  \overline{NX}]$


\begin{figure}[ht]
    \centering
		\scalebox{.7}{\incfig{ad-curve-no-color}}
    \label{fig:ad-curve-no-color}
\end{figure}

\begin{figure}[ht]
    \centering
		\scalebox{.7}{\incfig{ad-curve}}
    \label{fig:ad-curve}
\end{figure}


\begin{figure}[ht]
    \centering
		\scalebox{.7}{\incfig{ad-curve-y-ad-difference}}
    \label{fig:ad-curve-y-ad-difference}
\end{figure}
{\textbf {Conclusion:}}

Equilibrium appears if $ Y = AD $, e.g., point E. If firms produce more ($ Y>AD $),
i.e., $ Y > Y_0 $, there will be an increase in the inventory (UI). Hence they will
produce less in the next period (Move to the left). If the output level is less than
AD (on the left side of $ Y_0 $), there will be a decrease in the inventory. Firms
will produce more in the next period.

\subsubsection{Equilibrium condition}
In equilibrium, 
\begin{align*}
Y &= AD =  \overline{A} + cY\\
Y^{*}&= \frac{1}{1 - c} \overline{A}\\
&= \frac{1}{1 - c}[ \overline{C} - ( \overline{T}  -  \overline{TR}) +  \overline{I} +  \overline{G} +  \overline{NX}]
\end{align*}
where $ \frac{1}{1 - c} $ is called the {\textbf {spending multiplier}}, or just
{\textbf {multiplier}}.


\subsubsection{Multiplier}
Something you need to know:

If you purchase goods and services from Yan, the money you spent would be
considered as {\textbf {expenditure to you}}, but as {\textbf {income to Yan.}} 
If government spend $ \Delta G $ purchasing goods and services from person A, then

The expenditure for each of these people would be:
\begin{align*}
Gov &: G\\
Person-A&:G  \times MPC\\
B&: (G  \times MPC) \times MPC\\
C&: [(G  \times  MPC)  \times MPC] \times MPC\\
\cdots
\end{align*}


\begin{align*}
Expend iture =& G + G  \times MPC + (G  \times MPC) \times MPC \\
&  + [(G  \times  MPC)  \times MPC] \times MPC + \cdots\\
&= G(1 + MPC + MPC^{2} + MPC^{3} + \cdots + MPC^{n})
\end{align*}

In general
\begin{align*}
Y^{*}&= AD^{*} \frac{1}{1 - c} \overline{A}\\
\Delta AD &= \frac{1}{1 - c} \Delta  \overline{A} = \Delta Y
\end{align*}

\begin{figure}[ht]
    \centering
		\scalebox{.7}{\incfig{change-in-a}}
    \label{fig:change-in-a}
\end{figure}

\begin{figure}[ht]
    \centering
		\scalebox{.75}{\incfig{change-in-a-detailed}}
    \label{fig:change-in-a-detailed}
\end{figure}

{\textbf {Inferences:}}

While all over things are constant,
\begin{itemize}
\item a larger change in $ \Delta  \overline{A} $ would cause a larger increase in 
		income ($ \Delta Y $).
\item a higher MPC would cause a larger increase in income ($ \Delta Y $).
\begin{figure}[ht]
    \centering
    \incfig{low-c-high-c}
    \caption{low c high c}
    \label{fig:low-c-high-c}
\end{figure}
\end{itemize}





\subsubsection{Savings and Investment}
Income (Y) can be used for
\begin{equation*}
Y = C + S + T - TR
\end{equation*}
Aggregate demand: $ Y = C + I + G + NX $
\begin{align*}
C + I + G + NX &= C + S + T - TR\\
I &= S + (T - TR - G) - NX
\end{align*}
where $ (T - TR - G) $ represents gov't budget surplus.

{\textbf {If there is no G and NX}}, 
\begin{align*}
Y &= C + S\\
Y &= C + I\\
C + I&= C + S \\
I &= S
\end{align*}










\bibliographystyle{plainnat}
\bibliography{my_bib}

\end{document}

