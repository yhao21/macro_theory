
\documentclass[12pt]{article}
\usepackage{amsmath}
\DeclareMathOperator*{\argmin}{arg\,min} % thin space, limits underneath in displays
\DeclareMathOperator*{\argmax}{arg\,max} % thin space, limits underneath in displays
\newtheorem{thm}{Theorem}
\usepackage{amssymb}
\usepackage{amsfonts}
\usepackage{mathrsfs}
\usepackage{bm}
\usepackage{indentfirst}
\setlength{\parindent}{0em}
\usepackage[margin=1in]{geometry}
\usepackage{graphicx}
\usepackage{setspace}
\doublespacing
\usepackage[flushleft]{threeparttable}
\usepackage{booktabs,caption}
\usepackage{float}
\usepackage{graphicx}
\usepackage[sort,comma]{natbib}
\usepackage[hidelinks]{hyperref}

\usepackage{import}
\usepackage{xifthen}
\usepackage{pdfpages}
\usepackage{transparent}

\newcommand{\incfig}[1]{%
\def\svgwidth{\columnwidth}
\import{./figures/}{#1.pdf_tex}
}




\title{Exogenous Nominal Rigidity}
\author{}
\date{}


\begin{document}
\maketitle

A Walrasian model that is, a competitive model without any externalities,
asymmetric information, missing markets, or other imperfections.

\section{A baseline Case: Fixed Prices}
We assume the nominal price is fixed.

\subsection{Assumptions}
1. Time is discrete. Labor is the only input.
\begin{equation*}
F = F(L), \quad F'(\cdot ) > 0, \quad F''(\cdot ) \le 0
\end{equation*}

2. HHs obtain utility from 1) consumption, 2)holding real money balance, and disutility
from 3) working.
The HH's objective function,
\begin{equation*}
{\rm I\!{U}} = \sum\limits_{t = 0} ^\infty \beta^{t}	\left[ 
		U(C_{t}) + \Gamma \left( \frac{M_{t}}{P_{t}} \right)  - V(L_{t})
\right] , \quad \beta \in (0,1)
\end{equation*}
\begin{equation*}
U' > 0, U'' < 0, \Gamma' > 0, \Gamma '' < 0, V'>0, V'' < 0
\end{equation*}
$ U $ and $ \Gamma $ are in constant-relative-risk-aversion forms:
\begin{align*}
U(C_{t}) &= \frac{C_{t}^{1 - \theta}}{1 - \theta}, \quad \theta > 0\\
\Gamma \left( \frac{M_{t}}{P_{t}} \right)  &=
\frac{\left( \frac{M_{t}}{P_{t}} \right) ^{1 - \chi}}{1 - \chi}, \quad \chi > 0
\end{align*}

HHs obtain positive utility from holding cash because it allows them to purchase some
goods more easily.


3. Two assets:\\
1) Money, pays interest rate of 0.\\
2) Bond, pays interest rate of $ i_{t} $.

\begin{align*}
A_{t}&: \text{ HH's wealth at the start of period $ t $. }\\
W_{t}L_{t}&: \text{ labor income }\\
W_{t}&: \text{ nominal wage }\\
P_{t}C_{t}&: \text{ consumption expenditures }\\
M_{t}&: \text{ money in hand }
\end{align*}
HH spends the rest of income buying bonds. Hence, the bond it can hold from $ t $
to $ t + 1 $ is
\begin{equation*}
A_{t} + W_{t}L_{t} - P_{t}C_{t} - M_{t}
\end{equation*}

Therefore, the accumulation of wealth is
\begin{align*}
A_{t + 1} &= M_{t} + \text{ bonds and interest payment bring to $ t + 1 $ }\\
A_{t + 1}&= M_{t} + (A_{t} + W_{t}L_{t} - P_{t}C_{t} - M_{t})(1 + i_{t})
\end{align*}
HH takes $ W, P, i $ as give. It chooses $ C $ and $ M $ to maximize the utility.

The path of $ M $ is set by the central bank. Thus, although HHs view the path of $ M $
as something they choose, in general equilibrium, the path of $ M $ is exogenous,
and the path of $ i $ is determined endogenously.


\subsection{Utility maximization}
Rewrite asset accumulation process,
\begin{align*}
{\rm I\!{U}} &= \sum\limits_{t = 0} ^\infty \beta^{t}	\left[ 
		U(C_{t}) + \Gamma \left( \frac{M_{t}}{P_{t}} \right)  - V(L_{t})
\right] \\
A_{t + 1}&= M_{t} + (A_{t} + W_{t}L_{t} - P_{t}C_{t} - M_{t})(1 + i_{t})\\
C_{t}&= \left[ A_{t} + W_{t}L_{t} - M_{t} - \frac{A_{t + 1} - M_{t}}{1 + i_{t}} \right] 
\frac{1}{P_{t}}
\end{align*}
Plug in $ C_{t} $ into the lifetime utility function and differentiate it wrt 
$ A_{t + 1} $ to obtain the FOC.
\begin{align*}
\frac{\partial U }{\partial A_{t + 1} }&= 
\beta^{t}U'(C_{t})\left[ \frac{1}{P_{t}}\left(  - \frac{1}{1 + i_{t}} \right)  \right] 
 + \beta^{t + 1}U'(C_{t + 1})\frac{1}{P_{t + 1}} = 0\\
 U'(C_{t})\frac{1}{P_{t}}\frac{1}{1 + i_{t}} &= \beta U'(C_{t + 1})\frac{1}{P_{t + 1}}\\
 U'(C_{t})\frac{P_{t + 1}}{P_{t}}\frac{1}{1 + i_{t}}&= \beta U'(C_{t + 1}), \quad
 1 + \pi_{t} = \frac{P_{t + 1}}{P_{t}}\\
 U'(C_{t})\frac{1 + \pi_{t}}{1 + i_{t}}&= \beta U'(C_{t + 1}), \quad
 (1 + i_{t}) = (1 + r_{t})(1 + \pi_{t})\\
 U'(C_{t}) &= \beta(1 + r_{t})U'(C_{t + 1})\\
 C_{t}^{ - \theta}&= \beta(1 + r_{t})C_{t + 1}^{ - \theta}
\end{align*}

Take log
\begin{equation*}
\ln C_{t} = \ln C_{t + 1} - \frac{1}{\theta}\ln (1 + r_{t})\beta
\end{equation*}

Let's make some changes to this equation. Recall that the only use of ouput is for
consumption. And we normalize the number of HHs to 1. Then we can replace $ C $ by
$ Y $. In addition, because for small values of $ r $, $ \ln (1 + r)\approx r $. 
Replace $ \ln (1 + r) $ by $ r $. We receive this:
\begin{equation}
		\label{eqn:new keynesian IS}
\ln Y_{t} = a + \ln Y_{t + 1} - \frac{1}{\theta}r_{t}
\end{equation}
where $ a =  - \frac{1}{\theta}\ln \beta $.

Equation \eqref{eqn:new keynesian IS} is called {\textbf {the New Keynesian IS curve.}}







\bibliographystyle{plainnat}
\bibliography{my_bib}

\end{document}

