
\documentclass[12pt]{article}
\usepackage{amsmath}
\DeclareMathOperator*{\argmin}{arg\,min} % thin space, limits underneath in displays
\DeclareMathOperator*{\argmax}{arg\,max} % thin space, limits underneath in displays
\newtheorem{thm}{Theorem}
\usepackage{amssymb}
\usepackage{amsfonts}
\usepackage{mathrsfs}
\usepackage{bm}
\usepackage{indentfirst}
\setlength{\parindent}{0em}
\usepackage[margin=1in]{geometry}
\usepackage{graphicx}
\usepackage{setspace}
\doublespacing
\usepackage[flushleft]{threeparttable}
\usepackage{booktabs,caption}
\usepackage{float}
\usepackage{graphicx}
\usepackage[sort,comma]{natbib}
\usepackage[hidelinks]{hyperref}

\usepackage{import}
\usepackage{xifthen}
\usepackage{pdfpages}
\usepackage{transparent}

\newcommand{\incfig}[1]{%
\def\svgwidth{\columnwidth}
\import{./figures/}{#1.pdf_tex}
}




\title{Dynamic Stochastic General Equilibrium}
\author{}
\date{}


\begin{document}
\maketitle
Price changes are not only state dependent, they are partly time dependent.

Sections 7.2 through 7.4 then consider three baseline models
of time-dependent price adjustment:

the Fischer, or Fischer-Phelps-Taylor,
model; the Taylor model; and the Calvo model. 

All three models posit that prices (or wages) are set by multiperiod contracts or 
commitments. In each period, the contracts governing some fraction of prices expire and 
must be renewed; expiration is determined by the passage of time, not economic
developments.

{\textbf {The central result}} of the models is that multiperiod contracts
lead to gradual adjustment of the price level to nominal disturbances. As a
result, aggregate demand disturbances have persistent real effects.


{\textbf {Differences:}}
The Fischer model assumes that prices are predetermined but not fixed.
That is, when a multiperiod contract sets prices for several
periods, it can specify a different price for each period. 

In the Taylor and Calvo models, in contrast, prices are fixed: a contract must specify 
the same price each period it is in effect.


Section 7.5 then turns to two baseline models of state-dependent price
adjustment, the Caplin-Spulber and Danziger-Golosov-Lucas models
In both, the only barrier to price adjustment is a constant fixed cost. There are two
differences between the models.

{\textbf {First,}} money growth is always positive
in the Caplin-Spulber model, while the version of the Danziger-Golosov-
Lucas model we will consider assumes no trend money growth. 

{\textbf {Second,}} the
Caplin-Spulber model assumes no firm-specific shocks, while the Danziger-
Golosov-Lucas model includes them. Both models deliver strong results
about the effects of monetary disturbances, but for very different reasons.

Section 7.7 considers two more models of dynamic price adjustment: the Calvo-with-
indexation model and the Mankiw Reis model
These models are more complicated than the models of the earlier sections, but appear 
to have more hope of fitting key facts about inflation dynamics.


Section 7.8 presents
a complete DSGE model with nominal rigidity the canonical three-equation
new Keynesian model of Clarida, Galí, and Gertler.
But also like the baseline RBC model, the evidence for its key ingredients is weak,
In Section 7.9 that together the
ingredients make predictions about the macroeconomy that appear to be
almost embarrassingly incorrect.




\section{Common framework, dynamic new Keynesian models}
\subsection{HHs}
Utility from consumption and disutility from working.
\begin{align*}
		{\rm I\!{U}} &= \sum\limits_{t =0} ^\infty \beta^{t}[U(C_{t}) - V(L_{t})]	, \beta \in (0,1)\\
A_{t + 1}&= M_{t} + (A_{t} + W_{t}L_{t} - P_{t}C_{t} - M_{t})(1 + i_{t})\\
C_{t}&= \left[ A_{t} + W_{t}L_{t} - M_{t} - \frac{A_{t + 1} - M_{t}}{1 + i_{t}} \right] 
\frac{1}{P_{t}}\\
&= a_{t} + w_{t}L_{t} - \frac{i_{t}m_{t}}{1 + i_{t}} - \frac{a_{t + 1}}{1 + r_{t}}\\
U(C_{t})&= \frac{C_{t}^{1 - \theta}}{1 - \theta}, \quad \theta > 0\\
V(L_{t})&= \frac{B}{\gamma} L_{t}^{\gamma}, \quad B > 0, \quad \gamma > 1
\end{align*}
Price level, $ P_{t} $.

FOC wrt $ L_{t} $,
\begin{align*}
V'(L_{t}) &= U'(C_{t})\frac{W_{t}}{P_{t}}\\
\frac{V'(Y_{t})}{U'(Y_{t})}&= \frac{W_{t}}{P_{t}}
\end{align*}

Production function:
\begin{equation*}
Y_{i} = L_{i}, \quad i:= \text{ Firm }i
\end{equation*}
Production function is one-for-one and the only possible use of output is for 
consumption, in equilibrium, $ C_{t} = L_{t} = Y_{t} $

Plug in the utility function,
\begin{equation*}
\frac{W_{t}}{P_{t}} = BY_{t}^{\theta + \gamma - 1}
\end{equation*}
The new Keynesian IS curve holds:

Starting from FOC of $ C $:
\begin{align*}
C_{t}^{ - \theta}&= \beta(1 + r_{t})C_{t + 1}^{ - \theta}\\
\ln C_{t} &= \ln C_{t + 1} - \frac{1}{\theta}\ln (1 + r_{t})\beta\\
\ln Y_{t} &= a + \ln Y_{t + 1} - \frac{1}{\theta}r_{t}
\end{align*}
where $ a =  - \frac{1}{\theta}\ln \beta $, $ \ln (1 + r) \approx \ln r $.


\subsection{Firms}
Firm $ i $ faces the demand function:
\begin{equation*}
Y_{it} = Y_{t}\left( \frac{P_{it}}{P_{t}} \right) ^{ - \eta}
\end{equation*}

where $ \eta $ is the elasticity of substitution, $ \eta > 1 $.















\bibliographystyle{plainnat}
\bibliography{my_bib}

\end{document}

