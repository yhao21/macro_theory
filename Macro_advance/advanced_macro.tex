
\documentclass[12pt]{article}
\usepackage{amsmath}
\DeclareMathOperator*{\argmin}{arg\,min} % thin space, limits underneath in displays
\DeclareMathOperator*{\argmax}{arg\,max} % thin space, limits underneath in displays
\newtheorem{thm}{Theorem}
\usepackage{amssymb}
\usepackage{amsfonts}
\usepackage{mathrsfs}
\usepackage{bm}
\usepackage{indentfirst}
\setlength{\parindent}{0em}
\usepackage[margin=1in]{geometry}
\usepackage{graphicx}
\usepackage{setspace}
\doublespacing
\usepackage[flushleft]{threeparttable}
\usepackage{booktabs,caption}
\usepackage{float}
\usepackage{graphicx}
\usepackage[sort,comma]{natbib}
\usepackage[hidelinks]{hyperref}

\usepackage{import}
\usepackage{xifthen}
\usepackage{pdfpages}
\usepackage{transparent}

\newcommand{\incfig}[1]{%
\def\svgwidth{\columnwidth}
\import{./figures/}{#1.pdf_tex}
}




\title{}
\author{}
\date{}


\begin{document}

\section{Infinite Horizon}

Utility function in class:
\begin{equation*}
U = \sum\limits_{t = 0} ^\infty \frac{1}{(1 + \rho)^{t}}	u(C_{t}) = 
\sum\limits_{t = 0} ^\infty \beta^{t}u(C_{t})	
\end{equation*}
\begin{itemize}
		\item $ \rho $: marginal rate of time preference (impatience of HH), internal 
				discount rate. $ \beta = \frac{1}{1 + \rho} $. $ \beta \in (0,1) $, 
				$ \rho > 0 $.
		\item If $ \rho = 0 $, HH values consumption next period as much as this period.
				If $ \rho = 0 $, the life-time utility could goes to $ \infty  $. Not good.
		\item If $ \rho > 0 $, the larger the $ \rho $ is, the less value of 
				$ \frac{1}{1 + \rho} $, the more impatient the HH is (the utility of consumption
				in the next period is being weighted smaller and smaller). Consider a two 
				period case as an example.
\end{itemize}


Two period case:
\begin{equation*}
U = u(C_0) + \frac{1}{1 + \rho}u(C_{1})
\end{equation*}



\subsection{Math: continuous compounding}
Interest rate: $ r $. Initial asset: $ a $.

If we compound this annually, after $ t $ years, you receive
\begin{equation*}
\text{ you receive }: a(1 + r)^{t}
\end{equation*}

If we compound 2 times per year, your balance after the first year:
\begin{equation*}
a(1 + \frac{r}{2})^{2}
\end{equation*}
If we compound 2 times per year, after $ t $ years, you receive
\begin{equation*}
a \left( (1 + \frac{r}{2})^{2} \right) ^{t}
\end{equation*}

If we compound many times (n times) per year, after $ t $ years, you receive
\begin{equation*}
a \left( (1 + \frac{r}{n})^{n} \right) ^{t}
\end{equation*}

If we compound infinite times per year (n goes to $ \infty  $), it becomes a continuous
case, after $ t $ years, you receive
\begin{equation*}
\lim_{n \to \infty} a \left[  \left( 1 + \frac{r}{n} \right)^{n}  \right] ^{t}
\end{equation*}


\begin{align*}
\lim_{n \to \infty} \left( 1 + \frac{1}{n} \right) ^{n} &= e\\
\lim_{n \to \infty} \left( 1 + \frac{r}{n} \right) ^{n} &= e^{r}\\
\lim_{n \to \infty} \left[ \left( 1 + \frac{r}{n} \right)^{n}  \right] ^{t}&= e^{r t}
\end{align*}

If we compound the time preference $ \rho $ infinite times per year,
$ \frac{1}{1 + \rho} $
becomes
\begin{equation*}
 \frac{1}{ \lim_{n \to \infty}\left( 1 + \frac{\rho}{n} \right) ^{n}} = e^{-\rho}
\end{equation*}

Further, if we compound it infinite times per year for $ t $ years,
\begin{equation*}
\left(  \frac{1}{ \lim_{n \to \infty}\left( 1 + \frac{\rho}{n} \right) ^{n}}  \right) 
^{t} = e^{ - \rho t}
\end{equation*}

Now we finish the conversion below
\begin{equation*}
 \left( \frac{1}{1 + \rho} \right) ^{t}	
\rightarrow 
\left(  \frac{1}{ \lim_{n \to \infty}\left( 1 + \frac{\rho}{n} \right) ^{n}}  \right) 
^{t} = e^{ - \rho t}
\end{equation*}


Hence we can convert the discrete life-time utility function (LTU) to continuous LTU.
\begin{equation*}
\sum\limits_{t = 0} ^\infty \left( \frac{1}{1 + \rho} \right) ^{t} u(C_{t})
\rightarrow
\int_{t = 0}^{\infty } e^{ - \rho t}u(C(t))dt
\end{equation*}














%\bibliographystyle{plainnat}
%\bibliography{my_bib}

\end{document}

