
\documentclass[12pt]{article}
\usepackage{amsmath}
\DeclareMathOperator*{\argmin}{arg\,min} % thin space, limits underneath in displays
\DeclareMathOperator*{\argmax}{arg\,max} % thin space, limits underneath in displays
\newtheorem{thm}{Theorem}
\usepackage{amssymb}
\usepackage{amsfonts}
\usepackage{mathrsfs}
\usepackage{bm}
\usepackage{indentfirst}
\setlength{\parindent}{0em}
\usepackage[margin=1in]{geometry}
\usepackage{graphicx}
\usepackage{setspace}
\doublespacing
\usepackage[flushleft]{threeparttable}
\usepackage{booktabs,caption}
\usepackage{float}
\usepackage{graphicx}
\usepackage[sort,comma]{natbib}
\usepackage[hidelinks]{hyperref}

\usepackage{import}
\usepackage{xifthen}
\usepackage{pdfpages}
\usepackage{transparent}

\newcommand{\incfig}[1]{%
\def\svgwidth{\columnwidth}
\import{./figures/}{#1.pdf_tex}
}




\title{}
\author{}
\date{}


\begin{document}

Now introduce two models that resemble the Solow model but in which the dynamics of
economic aggregates are determined by decisions at the microeconomic level.

In this Note, we introduce the Ramsey-Cass-Koopmans model. It avoids all mkt 
imperfections and all inssues raised by heterogeneous HHs nd links among generations.

In the next Note, we introduce Overlapping-generations model developed by Diamond 
(1965).
The key difference of Diamond's model is that it assumes continual entry of new HHs
into the economy.






\section{Assumptions}

General Assumptions:

1. Labor and technology grow at an exogenous growth rate $ n $ and $ g $.

2. Competitive firms rent capital and hire labor to produce.

3. A fixed number of infinitely lived HHs.



\subsection{Firms}

Production Fn: CRTS
\begin{equation*}
Y = F(K,AL)
\end{equation*}


Capital accumulation is determined by output and consumption, same as the Solow model. 
However, we now assume {\textbf {No depreciation}}.

\begin{equation*}
\dot{K}(t) = Y(t) - \xi(t)
\end{equation*}

\begin{align*}
		\xi&: \text{ total consumption. }\\
		C &: \text{ consumption per person. }\\
		c &: \text{ consumption per effective labor. }
\end{align*}



\subsection{Households}

1. The size of each HH grows at rate $ n $. 

2. Each member of the HH supplies 1 unit of labor.

3. HH's income goes to consumption and saving.

\begin{align*}
H &: \text{ the number of HHs }\\
K(0)&: \text{ initial capital in the economy }\\
\frac{K(0)}{H} &: \text{ initial capital held by each HH }\\
C(t) &: \text{ consumption per person }\\
L(t) &: \text{ total population in the economy }\\
\frac{L(t)}{H} &: \text{ the number of members of each HH }\\
u(C(t))\frac{L(t)}{H} &: \text{ HH's utility at time $ t $ }\\
\rho &: \text{ discount rate. HH is impatient if $ \rho $ is large.}
\end{align*}

A discrete model:
\begin{equation*}
U = \sum\limits_{t = 0} ^\infty \left( \frac{1}{1 + \rho} \right) ^{t}
u(C(t))\frac{L(t)}{H}
\end{equation*}



Convert it to a continuous model:
\begin{equation*}
U = \int_{t = 0}^{\infty } e^{ - \rho t}u(C(t))\frac{L(t)}{H}dt
\end{equation*}
The utility fn:
\begin{equation*}
u(C(t)) = \frac{C(t)^{1 - \theta}}{1 - \theta}, \quad \theta > 0, \quad
\rho - n - (1 - \theta)g > 0
\end{equation*}
This function form is needed for the convergence (to the SS). It is known as 
{\textbf {constant-relative-risk-aversion}} (CRRA) utility.
The coefficient of relative risk aversion ($  - \frac{C u''}{u'} $) is $ \theta $,
and is independent from $ C $.

When $ \theta $ is smaller, MU falls more slowly (check $ u'' $) as $ C $ rises. So
HH is more willing to allow consumption to vary over time.

When $ \theta \rightarrow 1 $, it becomes $ u(C) = \ln C $

Utility will not explode to infinity because of  $ \rho - n - (1 - \theta)g > 0 $.






\section{Model}
\subsection{Firms}

No depreciation, so capital earn its MPK.
\begin{equation*}
r(t) = f'(k(t))
\end{equation*}

The MPL is 
\begin{equation*}
		\frac{\partial F(K, AL) }{\partial L } = A[f(k) - kf'(k)]
\end{equation*}
\noindent\fbox{%
\parbox{\textwidth}{%
Proof:
\begin{align*}
\frac{\partial F(K,AL) }{\partial L }&= \frac{\partial AL F(\frac{K}{AL},1) }
{\partial L }\\
&= A \left[ F(\frac{K}{AL},1) + L F'(\frac{K}{AL},1)( - \frac{K}{AL^{2}}) \right] \\
&= A \left( f(k) - Lf'(k)\frac{k}{L} \right), \quad k = \frac{K}{AL} \\
&= A [ f(k) - kf'(k)]
\end{align*}
}%
}\\



So the real wage, $ W(t) $, is 
\begin{equation*}
		W(t) = MPL = A(t)[f(k(t)) - k(t)f'(k(t))]
\end{equation*}

The wage per unit of effective labor is
\begin{equation*}
w(t) =f(k) - kf'(k) =  f(k(t)) - k(t)f'(k(t))
\end{equation*}


\subsection{Household Budget constraint}



To allow $ r $ changing over time, we define 
\begin{equation*}
R(t) = \int_{\tau = 0}^{t} r(\tau)d \tau
\end{equation*}

One unit of investment at time 0 yields $ e^{R(t)} $ units. It shows the continuously
compounding interest over the period [0,t].

The value of one unit of output at time $ t $ in terms of output at time 0 is
$ e^{ - R(t)} $. If $ r $ is constant, $ R(t) = r t $


 $  $.
\begin{align*}
\text{ Labor income for a HH }&:W(t)\frac{L(t)}{H}\\
\text{ Consumption expenditure for a HH }&: C(t)\frac{L(t)}{H}\\
\text{ initial wealth }&: \frac{1}{H} \text{ of total wealth at time 0 }
= \frac{K(0)}{H}
\end{align*}



{\textbf {HH BC}}:

\begin{equation*}
\int_{t = 0}^{\infty } e^{ - R(t)}C(t)\frac{L(t)}{H}d t \le 
\frac{K(0)}{H} + \int_{t = 0}^{\infty } e^{ - R(t)} W(t)\frac{L(t)}{H} d t
\end{equation*}










\end{document}

